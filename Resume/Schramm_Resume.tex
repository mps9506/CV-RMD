%!TEX TS-program = xelatex
%!TEX encoding = UTF-8 Unicode
% Awesome CV LaTeX Template for CV/Resume
%
% This template has been downloaded from:
% https://github.com/posquit0/Awesome-CV
%
% Author:
% Claud D. Park <posquit0.bj@gmail.com>
% http://www.posquit0.com
%
%
% Adapted to be an Rmarkdown template by Mitchell O'Hara-Wild
% 23 November 2018
%
% Template license:
% CC BY-SA 4.0 (https://creativecommons.org/licenses/by-sa/4.0/)
%
%-------------------------------------------------------------------------------
% CONFIGURATIONS
%-------------------------------------------------------------------------------
% A4 paper size by default, use 'letterpaper' for US letter
\documentclass[11pt, a4paper]{awesome-cv}

% Configure page margins with geometry
\geometry{left=1.4cm, top=.8cm, right=1.4cm, bottom=1.8cm, footskip=.5cm}

% Specify the location of the included fonts
\fontdir[fonts/]

% Color for highlights
% Awesome Colors: awesome-emerald, awesome-skyblue, awesome-red, awesome-pink, awesome-orange
%                 awesome-nephritis, awesome-concrete, awesome-darknight

\colorlet{awesome}{awesome-red}

% Colors for text
% Uncomment if you would like to specify your own color
% \definecolor{darktext}{HTML}{414141}
% \definecolor{text}{HTML}{333333}
% \definecolor{graytext}{HTML}{5D5D5D}
% \definecolor{lighttext}{HTML}{999999}

% Set false if you don't want to highlight section with awesome color
\setbool{acvSectionColorHighlight}{true}

% If you would like to change the social information separator from a pipe (|) to something else
\renewcommand{\acvHeaderSocialSep}{\quad\textbar\quad}

\def\endfirstpage{\newpage}

%-------------------------------------------------------------------------------
%	PERSONAL INFORMATION
%	Comment any of the lines below if they are not required
%-------------------------------------------------------------------------------
% Available options: circle|rectangle,edge/noedge,left/right

\name{Michael}{Schramm}

\address{3406 Judythe Ct. Bryan, TX 77803}

\mobile{910 232 3760}
\email{\href{mailto:mpschramm@gmail.com}{\nolinkurl{mpschramm@gmail.com}}}
\homepage{michaelpaulschramm.com}
\github{mps9506}
\twitter{mpschramm}

% \gitlab{gitlab-id}
% \stackoverflow{SO-id}{SO-name}
% \skype{skype-id}
% \reddit{reddit-id}


\usepackage{booktabs}

% Templates for detailed entries
% Arguments: what when with where why
\usepackage{etoolbox}
\def\detaileditem#1#2#3#4#5{%
\cventry{#1}{#3}{#4}{#2}{\ifx#5\empty\else{\begin{cvitems}#5\end{cvitems}}\fi}\ifx#5\empty{\vspace{-4.0mm}}\else\fi}
\def\detailedsection#1{\begin{cventries}#1\end{cventries}}

% Templates for brief entries
% Arguments: what when with
\def\briefitem#1#2#3{\cvhonor{}{#1}{#3}{#2}}
\def\briefsection#1{\begin{cvhonors}#1\end{cvhonors}}

\providecommand{\tightlist}{%
	\setlength{\itemsep}{0pt}\setlength{\parskip}{0pt}}

%------------------------------------------------------------------------------


%%%% BIBLIOGRAPHY
% Bibliography formatting

\usepackage[sorting=ynt,citestyle=authoryear,bibstyle=authoryear-comp,defernumbers=true,maxnames=20,giveninits=true, bibencoding=utf8, terseinits=true, uniquename=init,dashed=false,doi=false,isbn=false,natbib=true,backend=biber]{biblatex}

\DeclareFieldFormat{url}{\url{#1}}
\DeclareFieldFormat[article]{pages}{#1}
\DeclareFieldFormat[inproceedings]{pages}{\lowercase{pp.}#1}
\DeclareFieldFormat[incollection]{pages}{\lowercase{pp.}#1}
\DeclareFieldFormat[article]{volume}{\mkbibbold{#1}}
\DeclareFieldFormat[article]{number}{\mkbibparens{#1}}
\DeclareFieldFormat[article]{title}{\MakeCapital{#1}}
\DeclareFieldFormat[article]{url}{}
\DeclareFieldFormat[inproceedings]{title}{#1}
\DeclareFieldFormat{shorthandwidth}{#1}
\DeclareFieldFormat{extradate}{}

% No dot before number of articles
\usepackage{xpatch}
\xpatchbibmacro{volume+number+eid}{\setunit*{\adddot}}{}{}{}

% Remove In: for an article.
\renewbibmacro{in:}{%
  \ifentrytype{article}{}{%
  \printtext{\bibstring{in}\intitlepunct}}}

%\makeatletter
%\DeclareDelimFormat[cbx@textcite]{nameyeardelim}{\addspace}
%\makeatother

\setlength{\bibitemsep}{1.8pt}
\setlength{\bibhang}{.9cm}
%\renewcommand{\bibfont}{\fontsize{12}{14}}

\renewcommand*{\bibitem}{\addtocounter{papers}{1}\item \mbox{}\hskip-0.9cm\hbox to 0.9cm{\hfill\arabic{papers}.~\,}}
\defbibenvironment{bibliography}
{\list{}
  {\setlength{\leftmargin}{\bibhang}%
   \setlength{\itemsep}{\bibitemsep}%
   \setlength{\parsep}{\bibparsep}}}
{\endlist}
{\bibitem}

\renewcommand{\bibfont}{\normalfont\fontsize{10}{12.4}\selectfont}
% Counters for keeping track of papers
\newcounter{papers}

\DeclareSortingTemplate{ty}{
  \sort{
    \field{title}
  }
  \sort{
    \field{year}
  }
}
\DeclareBibliographyCategory{bib-C:/Users/michael.schramm/OneDrive - agnet.tamu.edu/Documents/GitHub/CV-RMD/Resume/Rpackages.bib-3032527}
\bibliography{C:/Users/michael.schramm/OneDrive - agnet.tamu.edu/Documents/GitHub/CV-RMD/Resume/Rpackages.bib}

\begin{document}

% Print the header with above personal informations
% Give optional argument to change alignment(C: center, L: left, R: right)
\makecvheader

% Print the footer with 3 arguments(<left>, <center>, <right>)
% Leave any of these blank if they are not needed
% 2019-02-14 Chris Umphlett - add flexibility to the document name in footer, rather than have it be static Curriculum Vitae
\makecvfooter
  {March 2020}
    {Michael Schramm~~~·~~~Resume}
  {\thepage}


%-------------------------------------------------------------------------------
%	CV/RESUME CONTENT
%	Each section is imported separately, open each file in turn to modify content
%------------------------------------------------------------------------------



\setbool{acvSectionColorHighlight}{false}

As a research professional providing expertise in water science, policy, and regulation, I leverage statistical and geospatial analysis to bridge the gaps between science and stakeholders. I invest my time in developing open source data tools and methods to ensure transparency and reproduciblity in the work I communicate to the public. Since 2016, I have authored nine papers or reports on water resources topics.

\hypertarget{skills}{%
\section{Skills}\label{skills}}

\textbf{Communication:} academic writing, extension and outreach programs, stakeholder facilitation, technical and non-technical reports

\textbf{Programatic:} grant writing, project management, proposal development

\textbf{Programming and Applications:} ArcGIS, Excel, git (limited), \texttt{R}, \texttt{Python}

\textbf{Water Management and Hydrologic Science:} statistical methods for water quality, TMDL development, water quality policy and regulations, watershed planning

\hypertarget{employment}{%
\section{Employment}\label{employment}}

\detailedsection{\detaileditem{Research Specialist III}{2019}{Texas Water Resources Institute, Texas A\&M AgriLife Research}{College Station, TX}{\item{Collaborate with researchers to design, plan, conduct, and coordinate water focused research projects.}\item{Facilitate stakeholder engagement to inform research priorities that address local research needs.}\item{Supervise undergraduate and/or graduate students and other technical or field staff involved in research.}\item{Represent TWRI at local, national, and international meetings to disseminate research findings and network with peers in the field.}\item{Prepare and write proposals to funding agencies; maintain financial accounts related to research projects.}}\detaileditem{Research Associate}{2016 - 2019}{Texas Water Resources Institute, Texas A\&M AgriLife Research}{College Station, TX}{\item{Provide technical support and stakeholder facilitation of watershed planning, TMDL, and I-Plan projects in collaboration with state water resource agencies.}\item{Develop, evaluate, and apply research and statistical methods for water resources planning.}\item{Collborated with state agencies and local stakeholders to publish watershed protection plans, TMDLs, Implementation Plans, and technical reports assessing indicator bacteria pollutant loads.}\item{Worked with Institute scientists and other entities to secure seven projects and \$1.3 million in research and water quality improvement funding.}}\detaileditem{Research Associate}{2014-2016}{Oak Ridge National Laboratory/Oak Ridge Associated Universities}{Oak Ridge, TN}{\item{Develop relational database and methods to assess environmental mitigation at U.S. hydropower facilities.}\item{Utilize statistical and geospatial methods to analyze data.}\item{Published three peer-reviewed journal articles, two technical reports, and one conference presentation on research findings related to mitigating environmental impacts of hydropower facilites.}}\detaileditem{Graduate Research Assistant}{2013-2014}{Center for Energy and Environmental Policy, University of Delaware}{Newark, DE}{\item{Responsible for interviews, data analysis, and developing policy reccomendations in two policy analysis reports delived to the state General Assembly.}}}

\hypertarget{education}{%
\section{Education}\label{education}}

\detailedsection{\detaileditem{Master of Energy and Environmental Policy}{2013}{University of Delaware}{Newark, DE}{\empty}\detaileditem{B.A. Environmental Studies}{2011}{University of North Carolina - Wilmington}{Wilmington, NC}{\empty}\detaileditem{B.S. Biology}{2004}{University of North Carolina - Wilmington}{Wilmington, NC}{\empty}}

\hypertarget{selected-publications}{%
\section{Selected Publications}\label{selected-publications}}

\hypertarget{software-r}{%
\section{Software (R)}\label{software-r}}

\defbibheading{bib-C:/Users/michael.schramm/OneDrive - agnet.tamu.edu/Documents/GitHub/CV-RMD/Resume/Rpackages.bib-3032527}{}\label{packagestart}
\addtocategory{bib-C:/Users/michael.schramm/OneDrive - agnet.tamu.edu/Documents/GitHub/CV-RMD/Resume/Rpackages.bib-3032527}{Rdartx,
Rechor,
Rtbrf,
Rwd4tx}
\newrefcontext[sorting=none]\setcounter{papers}{0}\pagebreak[3]\printbibliography[category=bib-C:/Users/michael.schramm/OneDrive - agnet.tamu.edu/Documents/GitHub/CV-RMD/Resume/Rpackages.bib-3032527,heading=none]\label{packageend}\setcounter{papers}{0}

\nocite{Rdartx,
Rechor,
Rtbrf,
Rwd4tx}

\end{document}
